\documentclass{beamer}
\usepackage{chronosys} 
\usepackage[utf8]{inputenc}
\usepackage{color}
\usepackage[acronym,toc]{glossaries} 
\usepackage{url} 
\usepackage{wrapfig}
\usepackage{subcaption}
\usepackage{xmpmulti}



\newcommand\ytl[2]{
	\parbox[b]{8em}{\hfill{\color{cyan}\bfseries\sffamily #1}~$\cdots\cdots$~}\makebox[0pt][c]{$\bullet$}\vrule\quad \parbox[c]{4.5cm}{\vspace{7pt}\color{red!40!black!80}\raggedright\sffamily #2\\[7pt]}\\[-3pt]} 

\title{BeagleBone Black project}
\author{Bc. et Bc. Jaroslav Svoboda, \\Andrea Vallejo, Diallo Elhadj Sadou}
\institute{Advanced Digital Architectures}
\date{2018}
\usetheme{Berlin} 


\begin{document}
\frame{\titlepage}
\begin{frame}
	\frametitle{Tasks}
		\begin{itemize}
			\item \textbf{Compile embedded Linux for BeagleBone Black}
			\item \textbf{Develop C application which uses}
			\begin{itemize}
				\item \textbf{TCS34725 RGB colour sensor}
				\item \textbf{L3GD20H 3-axis gyroscope}
			\end{itemize}
		\end{itemize}
\end{frame}
\begin{frame}
	\frametitle{Compiling Linux}
		\begin{itemize}
			\item Buildroot
			\begin{itemize}
				\item custom Linux kernel 3.12 with patches for Texas Instruments SoCs
				\item U-Boot 2016.03
				\item gdbserver for remote debugging
				\item openssh for remote connection
				\item set of elementary programs BusyBox
			\end{itemize}
		\end{itemize}
\end{frame}

\begin{frame}
	\frametitle{Files}
		\begin{itemize}
			\item FAT32 boot partition
			\begin{itemize}
				\item X-Loader - MLO
				\item U-boot binary - u-boot.bin
				\item Linux kernel - zImage
			\end{itemize}
		\item Linux partition
			\begin{itemize}
				\item filesystem - rootfs.ext2
			\end{itemize}
		\end{itemize}
\end{frame}

\begin{frame}
	\frametitle{Configuration}
		\begin{itemize}
			\item Allowed remote login for root user using password
					\begin{itemize}
						\item /etc/ssh/sshd\_config
						\item PermitRootLogin yes
						\item otherwise turn off!!!
					\end{itemize}
			\item Set static IP address
					\begin{itemize}
						\item /etc/network/interfaces
						\item iface eth0 inet static \\address 192.168.1.2 \\netmask 255.255.255.0	
					\end{itemize}
		\end{itemize}
\end{frame}

\begin{frame}
	\frametitle{Development}
		\begin{itemize}
			\item IDE Eclipse Oxygen from October 2017
			\item remote debugging 
			\item both devices in 192.168.1.0 subnet
			\item root user
		\end{itemize}
\end{frame}
\begin{frame}
\frametitle{Cross-Compilation Configuration}
\begin{itemize}
	\item Adding a line for cross-compiling tools to the PATH variable in the file “.profile”:\\
	PATH="/home/jaroslav/buildroot-2016-05/buildroot/output/host/usr/bin:\$PATH”
	\item Creation of C/C++ project: gyroRGB 
	\item Set the toolchain: Cross GCC
	\item Configuration: Debug [Activate]
	\item Prefix: arm-buildroot-linux-uclibcgnueabihf-
	\item Path: /home/jaroslav/buildroot-2016.05/output/host/usr/bin
\end{itemize}
\textbf{The compiling process is made in Linux x86 machine and the execution in ARM architecture (BBB)!}
\end{frame}
\begin{frame}
\frametitle{Development III}
Run \textbf{Debug Configuration} \\
Open a new \textbf{C/C++ Remote Application} and the code of the program to be debugged
\begin{itemize}
	\item Project name
	\item Build Configuration: Use Active 
	\item C/C++ Application: browse  executable
	\item Remote Absolute File Path for C/C++ Application
	\item Commands to execute before application: chmod +x location path of the application
	\item \textbf{GDB Debugger:} Location path of the \textbf{gdb}
	\item \textbf{GDB Command:} Location path of the \textbf{.gdbinit} file
	\item \textbf{Debugger Options:} Gdbserver Settings / Port number
\end{itemize}
\end{frame}
\begin{frame}
	\frametitle{Sensors}
	\begin{figure}[!ht]
		\centering
		\begin{subfigure}[b]{0.49\textwidth}
			\includegraphics[width=0.88\textwidth]{{"img/rgb"}.jpg}
			\caption{TCS34725 RGB colour sensor}
			\label{fig:rgb}
		\end{subfigure}
		\begin{subfigure}[b]{0.49\textwidth}
			\includegraphics[width=0.98\textwidth]{{"img/gyro"}.jpg}
			\caption{L3GD20H 3-axis gyroscope}
			\label{fig:gyro}
		\end{subfigure}
	\end{figure}
\end{frame}
\begin{frame}
\frametitle{Soldering}
\begin{figure}[!ht]
	\caption{BeagleBone Black with HAT with sensors}
	\label{fig:with_hat}
	\centering
	\includegraphics[width=0.78\textwidth]{{"img/withHATsideways"}.jpg}
\end{figure}

\end{frame}
	\begin{frame}
\frametitle{Introduction}
\begin{figure}[h]
	
\end{figure} 

\end{frame}
	\begin{frame}
\frametitle{Introduction}
\begin{figure}[h]
	
\end{figure} 

\end{frame}

\end{document}