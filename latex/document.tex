\documentclass[UKenglish,10pt,a4paper]{report}
\setlength\textwidth{145mm}
\setlength\textheight{247mm}
\setlength\oddsidemargin{5mm}
\setlength\evensidemargin{15mm}
\setlength\topmargin{0mm}
\setlength\headsep{0mm}
\setlength\headheight{0mm}
% \openright zařídí, aby následující text začínal na pravé straně knihy
\let\openright=\clearpage
\renewcommand{\baselinestretch}{1.0}%řádkování
\usepackage{booktabs}
\usepackage{multirow}
\usepackage{tabu}
\usepackage[hyphens]{url}
\usepackage{color}
\usepackage{listings}
\usepackage[utf8]{inputenc}
\usepackage{csquotes}
\usepackage[backend=biber]{biblatex}
\usepackage[pdftex]{graphicx}
\usepackage{epstopdf}
\usepackage{wrapfig}
\usepackage[section]{placeins}
\usepackage{amsfonts}
\usepackage{subcaption}
\usepackage{chronosys}
\usepackage{textcomp}
\usepackage{framed}
\usepackage{float}
\usepackage[czech, UKenglish]{babel}
\usepackage{multicol}
\usepackage{breakurl}
\usepackage[left]{eurosym}
\usepackage[binary-units]{siunitx}
\usepackage{makecell}
\usepackage{isodate}


\definecolor{mygreen}{rgb}{0,0.6,0}
\definecolor{mygray}{rgb}{0.5,0.5,0.5}
\definecolor{mymauve}{rgb}{0.58,0,0.82}



\newcommand\ytl[2]{
\parbox[b]{8em}{\hfill{\color{cyan}\bfseries\sffamily #1}~$\cdots\cdots$~}\makebox[0pt][c]{$\bullet$}\vrule\quad \parbox[c]{7.5cm}{\vspace{7pt}\color{red!40!black!80}\raggedright\sffamily #2\\[7pt]}\\[-3pt]}

\usepackage[T1]{fontenc}
\usepackage{amsmath}
%\usepackage[toc,page]{appendix}
\usepackage[acronym,toc]{glossaries} 

\usepackage[unicode,breaklinks]{hyperref}  % Musí být za všemi ostatními balíčky
\usepackage{bookmark}




%\usepackage{pdfpages}

\addbibresource{ref.bib}


%% Pokud tiskneme oboustranně:
% \documentclass[12pt,a4paper,twoside,openright]{report}
% \setlength\textwidth{145mm}
% \setlength\textheight{247mm}
% \setlength\oddsidemargin{15mm}
% \setlength\evensidemargin{0mm}
% \setlength\topmargin{0mm}
% \setlength\headsep{0mm}
% \setlength\headheight{0mm}
% \let\openright=\cleardoublepage

%% Pokud používáte csLaTeX (doporučeno):
%\usepackage[czech]{babel}
%% Pokud nikoliv:

\begingroup
    \makeatletter
    \catcode`\-=\active
    \AtBeginDocument{
    \def\@@@cmidrule[#1-#2]#3#4{\global\@cmidla#1\relax
        \global\advance\@cmidla\m@ne
        \ifnum\@cmidla>0\global\let\@gtempa\@cmidrulea\else
        \global\let\@gtempa\@cmidruleb\fi
        \global\@cmidlb#2\relax
        \global\advance\@cmidlb-\@cmidla
        \global\@thisrulewidth=#3
        \@setrulekerning{#4}
        \ifnum\@lastruleclass=\z@\vskip \aboverulesep\fi
        \ifnum0=`{\fi}\@gtempa
        \noalign{\ifnum0=`}\fi\futurenonspacelet\@tempa\@xcmidrule}
    }
\endgroup


%% Použité kódování znaků: obvykle latin2, cp1250 nebo utf8:
%\usepackage[cp1250]{inputenc}

%\usepackage[T1]{fontenc} %tohle by mělo udělat Times New Roman 
%% Ostatní balíčky

%\usepackage{mathtools}
%\usepackage{amsthm}

%% Balíček hyperref, kterým jdou vyrábět klikací odkazy v PDF,
%% ale hlavně ho používáme k uložení metadat do PDF (včetně obsahu).
%% POZOR, nezapomeňte vyplnit jméno práce a autora.
\hypersetup{pdftitle=Smart Lighting}
\hypersetup{pdfauthor={Bc. et Bc. Jaroslav Svoboda,Andrea Vallejo,Diallo Elhadj Sadou}, pdfkeywords={smart,lighting}}
%\hypersetup{hyperfootnotes=true}
\makeglossaries
%%% Drobné úpravy stylu

% Tato makra přesvědčují mírně ošklivým trikem LaTeX, aby hlavičky kapitol
% sázel příčetněji a nevynechával nad nimi spoustu místa. Směle ignorujte.
\makeatletter
\def\@makechapterhead#1{
  {\parindent \z@ \raggedright \normalfont
   \Huge\bfseries \thechapter. #1
   \par\nobreak
   \vskip 20\p@
}}
\def\@makeschapterhead#1{
  {\parindent \z@ \raggedright \normalfont
   \Huge\bfseries #1
   \par\nobreak
   \vskip 20\p@
}}
\makeatother

% Toto makro definuje kapitolu, která není očíslovaná, ale je uvedena v obsahu.
\def\chapwithtoc#1{
\chapter*{#1}}
\renewcommand{\familydefault}{\rmdefault}
\begin{document}
\selectlanguage{UKenglish}
% Trochu volnější nastavení dělení slov, než je default.
\lefthyphenmin=2
\righthyphenmin=2
\sisetup{range-phrase=--}
%%% Titulní strana práce
\begin{titlepage}


\pagestyle{empty}
\begin{center}

\large

TECHNICAL UNIVERSITY OF MADRID

\medskip

SCHOOL OF TELECOMMUNICATIONS SYSTEMS AND ENGINEERING

\vfill

{\bf\Large Semester Project}

\vfill
\centerline{\mbox{\includegraphics[width=60mm]{img/logo2.eps}\qquad{\includegraphics[width=60mm]{img/logo.eps}}}}
			
\vfill
\vspace{5mm}

{\LARGE Bc. et Bc. Jaroslav Svoboda}

{\LARGE Andrea Vallejo}

{\LARGE Diallo Elhadj Sadou}

\vspace{10mm}

% Název práce přesně podle zadání
{\LARGE\bfseries } BeagleBone Black Project

\vfill

% Název katedry nebo ústavu, kde byla práce oficiálně zadána
% (dle Organizační struktury MFF UK)
ADVANCED DIGITAL ARCHITECTURES

\vfill

\begin{tabular}{rl}

Lecturer: & Eduardo Juárez  \\
\noalign{\vspace{2mm}}
Subject coordinator: & Mariano Ruiz \\
\noalign{\vspace{2mm}}
Degree programme: & Master in Systems and Services Engineering for the Information society \\
\end{tabular}

\vfill

% Zde doplňte rok
Madrid 2018

\end{center}
\end{titlepage}
\newpage

%%% Následuje vevázaný list -- kopie podepsaného "Zadání bakalářské práce".
%%% Toto zadání NENÍ součástí elektronické verze práce, nescanovat.


\openright

\noindent



%%% Strana s automaticky generovaným obsahem bakalářské práce. U matematických
%%% prací je přípustné, aby seznam tabulek a zkratek, existují-li, byl umístěn
%%% na začátku práce, místo na jejím konci.



\openright
\pagestyle{plain}
\setcounter{page}{1}
\tableofcontents

\lstset{language=C,
	backgroundcolor=\color{white},   % choose the background color
	basicstyle=\footnotesize,        % size of fonts used for the code
	breaklines=true,                 % automatic line breaking only at whitespace
	captionpos=b,                    % sets the caption-position to bottom
	commentstyle=\color{mygreen},    % comment style
	escapeinside={\%*}{*)},          % if you want to add LaTeX within your code
	keywordstyle=\color{blue},       % keyword style
	stringstyle=\color{mymauve},     % string literal style
}

\chapter{Embedded Linux}
Whole project was done using Ubuntu GNOME 14.04.5 LTS Trusty Tahr, running in VMware Workstation 14.1.1, hosted by Kali Linux 2017.2. First we had to compile embedded Linux using Buildroot maintained by Peter Korsgaard, a tool simplifying and automating the building process of bootable Linux system for embedded solutions using cross-compilation for architectural independence. When using Buildroot we followed instructions given us in the document for lab. We set parameters accordingly to our build target BeagleBone Black which uses Texas Instruments Sitara AM3358, an ARMv7-A processor with one Cortex-A8  core. Our build uses custom Linux kernel 3.12 which includes patches for Texas Instruments SoCs. Linux is booted using U-Boot 2016.03 for processors from AM335x series. Build includes gdbserver for remote debugging, openssh for remote connection and set of elementary programs BusyBox. Two users were created: prdel and root both with password ada.
\begin{figure}[!ht]
	\centering
	\begin{subfigure}[b]{0.49\textwidth}
		\includegraphics[width=0.98\textwidth]{{"img/top"}.jpg}
		\caption{BeagleBone Black top}
		\label{fig:top}
	\end{subfigure}
	\begin{subfigure}[b]{0.49\textwidth}
		\includegraphics[width=0.98\textwidth]{{"img/bottom"}.jpg}
		\caption{BeagleBone Black bottom}
		\label{fig:bottom}
	\end{subfigure}
\end{figure}

After successful compilation we created two partitions on SD card: FAT32 boot partition which includes the X-Loader (MLO), U-boot binary (u-boot.bin), Linux kernel (zImage) and device tree binaries (*.dtb), and Linux filesystem partition which is created from rootfs.ext2 file. This was one of the tricky parts and required several attempts.

When BBB successfully booted, several settings was changed using root user. Firstly, in settings of ssh deamon, file /etc/ssh/sshd\_config, root user was allowed to login with password.
\begin{lstlisting}[language=bash,frame=single]
PermitRootLogin yes
\end{lstlisting}
 Secondly, device was configured, file /etc/network/interface, to use static IP address for direct connection to VM in order to use remote debugging.
\begin{lstlisting}[language=bash,frame=single]
iface eth0 inet static
	address 192.168.1.2
	netmask 255.255.255.0
\end{lstlisting} 
 We had to rewrite the filesystem several times because it was corrupted.

\chapter{Application development and connecting of sensors}
First step in connecting the TCS34725 RGB colour sensor and the L3GD20H 3-axis gyroscope was soldiering them onto HAT for practical reasons. Sensors use ground aka pins 1 and 2, 3.3 V from pins 3 and 4 for power, pins 17 and 19 for I\textsuperscript{2}C clock line and pins 18 and 20 for I\textsuperscript{2}C data line.
\begin{figure}[!ht]
	\caption{BeagleBone Black with HAT with sensors}
	\label{fig:with_hat}
	\centering
	\includegraphics[width=0.98\textwidth]{{"img/withHATsideways"}.jpg}
\end{figure}

\begin{figure}[!ht]
	\centering
	\begin{subfigure}[b]{0.49\textwidth}
		\includegraphics[width=0.98\textwidth]{{"img/rgb"}.jpg}
		\caption{TCS34725 RGB colour sensor}
		\label{fig:rgb}
	\end{subfigure}
	\begin{subfigure}[b]{0.49\textwidth}
		\includegraphics[width=0.98\textwidth]{{"img/gyro"}.jpg}
		\caption{L3GD20H 3-axis gyroscope}
		\label{fig:gyro}
	\end{subfigure}
\end{figure}

For development in C we used Eclipse Oxygen from October 2017. We set up remote debugging and cross compiling according to instructions for lab document. We used direct connection over Ethernet for remote debugging when both devices were in 192.168.1.0 subnet. Developing and remote debugging in Eclipse Oxygen was not without complications and is definitely not stable and reliable.


In application we first initialize the I\textsuperscript{2}C bus and open it for reading to find out if the I\textsuperscript{2}C is available. The TCS34725 sensor address is $0x29$ as mentioned in \cite{rgb}. 

\begin{lstlisting}[language=C,frame=single]
// Create I2C bus
int file;
int file2;
char *bus = "/dev/i2c-1";
char *bus2 = "/dev/i2c-2";

if ((file = open(bus, O_RDWR)) < 0)
{
	printf("Failed to open the bus. \n");
	exit(1);
}
else
{ //the bus of RGB sensor was successfully opened
	printf("It's alive! \n");
	
	
	
	// Get I2C device, TCS34725 I2C address is 0x29(41)
	ioctl(file, I2C_SLAVE, 0x29);
\end{lstlisting} 

According to Linux description of the command the ioctl() system call manipulates the underlying device parameters of special files. In particular, many operating characteristics of character special files (e.g. terminals) may be controlled with  ioctl() requests.

We refer to the table in \cite{rgb} to configure the table config we have created as you can see the address on the top and the role of each register. With this method we initialize other registers.

\begin{lstlisting}[language=C,frame=single]
// Select enable register(0x80)
// Power ON, RGBC enable, wait time disable(0x03)
char config[2] = {0};
config[0] = 0x80;
config[1] = 0x03;
write(file, config, 2);

// Select ALS time register(0x81)
// Atime = 700 ms(0x00)
config[0] = 0x81;
config[1] = 0x00;
write(file, config, 2);
// Select Wait Time register(0x83)
// WTIME : 2.4ms(0xFF)
config[0] = 0x83;
config[1] = 0xFF;
write(file, config, 2);
// Select control register(0x8F)k
// AGAIN = 1x(0x00)
config[0] = 0x8F;
config[1] = 0x00;
write(file, config, 2);
sleep(1);
\end{lstlisting} 

For the last part we just needed to know the formula of the luminance with the weight of the colour luminance for each colour.

\begin{lstlisting}[language=C,frame=single]
// Calculate luminance
float luminance = (-0.32466) * (red) + (1.57837) * (green) + (-0.73191) * (blue);
if(luminance < 0)
{
	luminance = 0;
}

// Output data to screen
printf("Red color luminance : %d lux \n", red);
printf("Green color luminance : %d lux \n", green);
printf("Blue color luminance : %d lux \n", blue);
printf("IR  luminance : %d lux \n", cData);
printf("Ambient Light Luminance : %.2f lux \n", luminance);
}
\end{lstlisting} 

Project, application, presentation and this document, is available in Git repository \url{https://github.com/multiflexi/adamadrina.git}

\chapter{Code}

\begin{lstlisting}[language=C]
// Distributed with a free-will license.
// Use it any way you want, profit or free, provided it fits in the licenses of its associated works.
// adapted by Andrea Vallejo, Jaroslav Svoboda and Diallo

#include <stdio.h>
#include <stdlib.h>
#include <linux/i2c-dev.h>
#include <sys/ioctl.h>
#include <fcntl.h>


int main()
{
// Create I2C bus
int file;
int file2;
char *bus = "/dev/i2c-1";
char *bus2 = "/dev/i2c-2";

if ((file = open(bus, O_RDWR)) < 0)
{
printf("Failed to open the bus. \n");
exit(1);
}
else
{ //the bus of RGB sensor was successfully opened
printf("It's alive! \n");



// Get I2C device, TCS34725 I2C address is 0x29(41)
ioctl(file, I2C_SLAVE, 0x29);

// Select enable register(0x80)
// Power ON, RGBC enable, wait time disable(0x03)
char config[2] = {0};
config[0] = 0x80;
config[1] = 0x03;
write(file, config, 2);

// Select ALS time register(0x81)
// Atime = 700 ms(0x00)
config[0] = 0x81;
config[1] = 0x00;
write(file, config, 2);
// Select Wait Time register(0x83)
// WTIME : 2.4ms(0xFF)
config[0] = 0x83;
config[1] = 0xFF;
write(file, config, 2);
// Select control register(0x8F)k
// AGAIN = 1x(0x00)
config[0] = 0x8F;
config[1] = 0x00;
write(file, config, 2);
sleep(1);

// Read 8 bytes of data from register(0x94)
// cData lsb, cData msb, red lsb, red msb, green lsb, green msb, blue lsb, blue msb
char reg[1] = {};
write(file, reg, 1);
char data[8] = {0};
if(read(file, data, 8) != 8)
{
printf("Erorr : Input/output Erorr \n");
}
else
{
// Convert the data
int cData = (data[1] * 256 + data[0]);
int red = (data[3] * 256 + data[2]);
int green = (data[5] * 256 + data[4]);
int blue = (data[7] * 256 + data[6]);

// Calculate luminance
float luminance = (-0.32466) * (red) + (1.57837) * (green) + (-0.73191) * (blue);
if(luminance < 0)
{
luminance = 0;
}

// Output data to screen
printf("Red color luminance : %d lux \n", red);
printf("Green color luminance : %d lux \n", green);
printf("Blue color luminance : %d lux \n", blue);
printf("IR  luminance : %d lux \n", cData);
printf("Ambient Light Luminance : %.2f lux \n", luminance);
}


//STARTS GYRO
if ((file2 = open(bus2, O_RDWR)) < 0)
{
printf("Failed to open the second bus. \n");
exit(1);
}
else
{ //the bus of gyro was successfully open
printf("It's alive!...twice... \n");

printf("the value returned is %d \n",ioctl(file2, I2C_SLAVE, 0x6B));

// Enable X, Y, Z-Axis and disable Power down mode(0x0F)
char config[2] = {0};
config[0] = 0x20;
config[1] = 0x0F;
write(file2, config, 2);
// Full scale range, 2000 dps(0x30)
config[0] = 0x23;
config[1] = 0x30;
write(file2, config, 2);
sleep(1);

// Read 6 bytes of data
// lsb first
// Read xGyro lsb data from register(0x28)
char reg[1] = {0x28};
write(file2, reg, 1);
char datai[1] = {0};
if(read(file2, datai, 1) != 1)
{
printf("Error : Input/Output Error \n");
exit(1);
}
char data_0 = datai[0];

// Read xGyro msb data from register(0x29)
reg[0] = 0x29;
write(file2, reg, 1);
read(file2, datai, 1);
char data_1 = datai[0];

// Read yGyro lsb data from register(0x2A)
reg[0] = 0x2A;
write(file2, reg, 1);
read(file2, datai, 1);
char data_2 = datai[0];

// Read yGyro msb data from register(0x2B)
reg[0] = 0x2B;
write(file2, reg, 1);
read(file2, datai, 1);
char data_3 = datai[0];

// Read zGyro lsb data from register(0x2C)
reg[0] = 0x2C;
write(file2, reg, 1);
read(file2, datai, 1);
char data_4 = datai[0];

// Read zGyro msb data from register(0x2D)
reg[0] = 0x2D;
write(file2, reg, 1);
read(file2, datai, 1);
char data_5 = datai[0];

// Convert the data
int xGyro = (data_1 * 256 + data_0);
if(xGyro > 32767)
{
xGyro -= 65536;
}

int yGyro = (data_3 * 256 + data_2);
if(yGyro > 32767)
{
yGyro -= 65536;
}

int zGyro = (data_5 * 256 + data_4);
if(zGyro > 32767)
{
zGyro -= 65536;
}
// Output data to screen
printf("Rotation in X-Axis : %d \n", xGyro);
printf("Rotation in Y-Axis : %d \n", yGyro);
printf("Rotation in Z-Axis : %d \n", zGyro);
}


exit(0);

}
}
\end{lstlisting} 
\renewcommand{\acronymname}{List of acronyms}
\printglossaries



%\bibliographystyle{plain} 
\printbibliography
%%% Přílohy k bakalářské práci, existují-li (různé dodatky jako výpisy programů,
%%% diagramy apod.). Každá příloha musí být alespoň jednou odkazována z vlastního
%%% textu práce. Přílohy se číslují.
\openright
\end{document}