\documentclass[UKenglish,10pt,a4paper]{report}
\setlength\textwidth{145mm}
\setlength\textheight{247mm}
\setlength\oddsidemargin{5mm}
\setlength\evensidemargin{15mm}
\setlength\topmargin{0mm}
\setlength\headsep{0mm}
\setlength\headheight{0mm}
% \openright zařídí, aby následující text začínal na pravé straně knihy
\let\openright=\clearpage
\renewcommand{\baselinestretch}{1.0}%řádkování
\usepackage{booktabs}
\usepackage{multirow}
\usepackage{tabu}
\usepackage[hyphens]{url}
\usepackage{color}
\usepackage{listings}
\usepackage[utf8]{inputenc}
\usepackage{csquotes}
\usepackage[backend=biber]{biblatex}
\usepackage[pdftex]{graphicx}
\usepackage{epstopdf}
\usepackage{wrapfig}
\usepackage[section]{placeins}
\usepackage{amsfonts}
\usepackage{subcaption}
\usepackage{chronosys}
\usepackage{textcomp}
\usepackage{framed}
\usepackage{float}
\usepackage[czech, UKenglish]{babel}
\usepackage{multicol}
\usepackage{breakurl}
\usepackage[left]{eurosym}
\usepackage[binary-units]{siunitx}
\usepackage{makecell}
\usepackage{isodate}





\newcommand\ytl[2]{
\parbox[b]{8em}{\hfill{\color{cyan}\bfseries\sffamily #1}~$\cdots\cdots$~}\makebox[0pt][c]{$\bullet$}\vrule\quad \parbox[c]{7.5cm}{\vspace{7pt}\color{red!40!black!80}\raggedright\sffamily #2\\[7pt]}\\[-3pt]}

\usepackage[T1]{fontenc}
\usepackage{amsmath}
%\usepackage[toc,page]{appendix}
\usepackage[acronym,toc]{glossaries} 

\usepackage[unicode,breaklinks]{hyperref}  % Musí být za všemi ostatními balíčky
\usepackage{bookmark}




%\usepackage{pdfpages}

\addbibresource{ref.bib}


%% Pokud tiskneme oboustranně:
% \documentclass[12pt,a4paper,twoside,openright]{report}
% \setlength\textwidth{145mm}
% \setlength\textheight{247mm}
% \setlength\oddsidemargin{15mm}
% \setlength\evensidemargin{0mm}
% \setlength\topmargin{0mm}
% \setlength\headsep{0mm}
% \setlength\headheight{0mm}
% \let\openright=\cleardoublepage

%% Pokud používáte csLaTeX (doporučeno):
%\usepackage[czech]{babel}
%% Pokud nikoliv:

\begingroup
    \makeatletter
    \catcode`\-=\active
    \AtBeginDocument{
    \def\@@@cmidrule[#1-#2]#3#4{\global\@cmidla#1\relax
        \global\advance\@cmidla\m@ne
        \ifnum\@cmidla>0\global\let\@gtempa\@cmidrulea\else
        \global\let\@gtempa\@cmidruleb\fi
        \global\@cmidlb#2\relax
        \global\advance\@cmidlb-\@cmidla
        \global\@thisrulewidth=#3
        \@setrulekerning{#4}
        \ifnum\@lastruleclass=\z@\vskip \aboverulesep\fi
        \ifnum0=`{\fi}\@gtempa
        \noalign{\ifnum0=`}\fi\futurenonspacelet\@tempa\@xcmidrule}
    }
\endgroup


%% Použité kódování znaků: obvykle latin2, cp1250 nebo utf8:
%\usepackage[cp1250]{inputenc}

%\usepackage[T1]{fontenc} %tohle by mělo udělat Times New Roman 
%% Ostatní balíčky

%\usepackage{mathtools}
%\usepackage{amsthm}

%% Balíček hyperref, kterým jdou vyrábět klikací odkazy v PDF,
%% ale hlavně ho používáme k uložení metadat do PDF (včetně obsahu).
%% POZOR, nezapomeňte vyplnit jméno práce a autora.
\hypersetup{pdftitle=Smart Lighting}
\hypersetup{pdfauthor={Bc. et Bc. Jaroslav Svoboda,Andrea Vallejo,Diallo Elhadj Sadou}, pdfkeywords={smart,lighting}}
%\hypersetup{hyperfootnotes=true}
\makeglossaries
%%% Drobné úpravy stylu

% Tato makra přesvědčují mírně ošklivým trikem LaTeX, aby hlavičky kapitol
% sázel příčetněji a nevynechával nad nimi spoustu místa. Směle ignorujte.
\makeatletter
\def\@makechapterhead#1{
  {\parindent \z@ \raggedright \normalfont
   \Huge\bfseries \thechapter. #1
   \par\nobreak
   \vskip 20\p@
}}
\def\@makeschapterhead#1{
  {\parindent \z@ \raggedright \normalfont
   \Huge\bfseries #1
   \par\nobreak
   \vskip 20\p@
}}
\makeatother

% Toto makro definuje kapitolu, která není očíslovaná, ale je uvedena v obsahu.
\def\chapwithtoc#1{
\chapter*{#1}}
\renewcommand{\familydefault}{\rmdefault}
\begin{document}
\selectlanguage{UKenglish}
% Trochu volnější nastavení dělení slov, než je default.
\lefthyphenmin=2
\righthyphenmin=2
\sisetup{range-phrase=--}
%%% Titulní strana práce
\begin{titlepage}


\pagestyle{empty}
\begin{center}

\large

TECHNICAL UNIVERSITY OF MADRID

\medskip

SCHOOL OF TELECOMMUNICATIONS SYSTEMS AND ENGINEERING

\vfill

{\bf\Large Semester Project}

\vfill
\centerline{\mbox{\includegraphics[width=60mm]{img/logo2.eps}\qquad{\includegraphics[width=60mm]{img/logo.eps}}}}
			
\vfill
\vspace{5mm}

{\LARGE Bc. et Bc. Jaroslav Svoboda}

{\LARGE Andrea Vallejo}

{\LARGE Diallo Elhadj Sadou}

\vspace{10mm}

% Název práce přesně podle zadání
{\LARGE\bfseries } BeagleBone Black Project

\vfill

% Název katedry nebo ústavu, kde byla práce oficiálně zadána
% (dle Organizační struktury MFF UK)
ADVANCED DIGITAL ARCHITECTURES

\vfill

\begin{tabular}{rl}

Lecturer: & Eduardo Juárez  \\
\noalign{\vspace{2mm}}
Subject coordinator: & Mariano Ruiz \\
\noalign{\vspace{2mm}}
Degree programme: & Master in Systems and Services Engineering for the Information society \\
\end{tabular}

\vfill

% Zde doplňte rok
Madrid 2018

\end{center}
\end{titlepage}
\newpage

%%% Následuje vevázaný list -- kopie podepsaného "Zadání bakalářské práce".
%%% Toto zadání NENÍ součástí elektronické verze práce, nescanovat.


\openright

\noindent



%%% Strana s automaticky generovaným obsahem bakalářské práce. U matematických
%%% prací je přípustné, aby seznam tabulek a zkratek, existují-li, byl umístěn
%%% na začátku práce, místo na jejím konci.



\openright
\pagestyle{plain}
\setcounter{page}{1}
\tableofcontents


\chapter{Embedded Linux}
Whole project was done using Ubuntu GNOME 14.04.5 LTS Trusty Tahr, running in VMware Workstation 14.1.1, hosted by Kali Linux 2017.2. First we had to compile embedded Linux using Buildroot maintained by Peter Korsgaard, a tool simplifying and automating the building process of bootable Linux system for embedded solutions using cross-compilation for architectural independence. When using Buildroot we followed instructions given us in the document for lab. We set parameters accordingly to our build target BeagleBone Black which uses Texas Instruments Sitara AM3358, an ARMv7-A processor with one Cortex-A8  core. Our build uses custom Linux kernel 3.12 which includes patches for Texas Instruments SoCs. Linux is booted using U-Boot 2016.03 for processors from AM335x series. Build includes gdbserver for remote debugging, openssh for remote connection and set of elementary programs BusyBox.

After successful compilation we created two partitions on SD card: FAT32 boot partition which includes the X-Loader (MLO), U-boot binary (u-boot.bin), Linux kernel (zImage) and device tree binaries (*.dtb), and Linux filesystem partition which is created from rootfs.ext2 file. This was one of the tricky parts and required several attempts.

When BBB successfully booted, several settings was changed using root user. Firstly, in settings of ssh deamon root user was allowed to login with password. Secondly, device was configured to use static IP address for direct connection to VM in order to use remote debugging.
\chapter{Application and sensors}
First step in connecting the TCS34725 RGB colour sensor and the  L3GD20H 3-axis gyroscope was soldiering them onto HAT for practical reasons. Sensors use ground aka pins 1 and 2, 3.3 V from pins 3 and 4 for power, pins 17 and 19 for I\textsuperscript{2}C clock line and pins 18 and 20 for I\textsuperscript{2}C data line.

For development we used Eclipse Oxygen from October 2017. We set up remote debugging and cross compiling according to instructions for lab document.

\renewcommand{\acronymname}{List of acronyms}
\printglossaries



%\bibliographystyle{plain} 
\printbibliography
%%% Přílohy k bakalářské práci, existují-li (různé dodatky jako výpisy programů,
%%% diagramy apod.). Každá příloha musí být alespoň jednou odkazována z vlastního
%%% textu práce. Přílohy se číslují.
\openright
\end{document}